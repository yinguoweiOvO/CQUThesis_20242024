% !TeX encoding = UTF-8
%% \textbf{重庆大学}通用毕业论文\LaTeXe{}模板
%%% 使用前请先阅读使用文档和用户协议,内有详细介绍。Happy Texing! :)
%% =======================================================
\documentclass%
	[type=master, bilinguallist=apart, 
	printmode=oneside]{cquthesis}%
% 可用选项:
% type=[bachelor|master|doctor],      % 必选,毕业论文类型,以下项目不填时为默认
% liberalformat,                      % 可选,仅适用本科生,使用文学类论文标题格式,默认未打开
% proffesionalmaster=[true|false],    % 可选,仅适用研究生,是(true)否(false)专业硕士,默认为否
% printmode=[oneside|twoside|auto],	  % 可选,论文打印方式,默认采用auto按页数要求自动判定
% openany,|openright,                 % 可选,双面打印时每章的第一页仅右页开启,默认右页开启(openright)
% bilinguallist=[off|combined|apart], % 可选,图录表录等分别按双语题注混编(combined),分开编录(apart),默认关(off)
% blindtrail,                         % 可选,盲审模式,开启后封面姓名和致谢部分会隐藏,详情请参阅用户文档,默认关
% draft,                              % 写作期间可选,不渲染图片,关闭外围功能,加快预览速度,默认未开启

% 请在cquthesis.sty文件中定义其他会用到的宏包和自己的变量
% 这样可以防止main.tex太过臃肿。
\usepackage{cquthesis}

% 定义所有的图片文件在 figures 子目录下
\graphicspath{{figures/}}

% 定义数字圆
\usepackage{tikz}
\newcommand*\circled[1]{\tikz[baseline=(char.base)]{
            \node[shape=circle,draw,inner sep=1pt] (char) {\small #1};}}

%*** 写作时,使用这个命令只渲染你想查看的部分,提升工作效率,定稿时注释掉整行
% \includeonly{contents/introduction}
% \includeonly{contents/related_work}
% \includeonly{contents/MGRCL}
% \includeonly{contents/S2V}


\begin{document}

\cqusetup{
%	************	注意	************
%	* 1. \cqusetup{}中不能出现全空的行,如果需要全空行请在行首注释
%	* 2. 不需要的配置信息可以放心地坐视不理、留空、删除或注释(都不会有影响)
%	*
%	********************************
% ===================
%	论文的中英文题目
% ===================
  ctitle = {基于多元关系建模的少样本分类算法研究},
  etitle = {Research on Few-Shot Classification Algorithms Based on Multivariate Relationship Modeling},
% ===================
% 作者部分的信息
% \secretize{}为盲审标记点,在打开盲审开关时内容会自动被替换为***输出,盲审开关默认关闭
% ===================
  cauthor = \secretize{尹国伟},	% 你的姓名,以下每项都以英文逗号结束
%   eauthor = \secretize{Guowei~Yin},	% 姓名拼音,~代表不会断行的空格
%   studentid = \secretize{},	% 仅本科生,学号
%   csupervisor = \secretize{黄~~~晟~~~~~教授},	% 导师的姓名
%   esupervisor = \secretize{{Prof.~Sheng Huang}},	% 导师的姓名拼音
%   cassistsupervisor = \secretize{}, % 本科生可选,助理指导教师姓名,不用时请留空为{}
%   cextrasupervisor = \secretize{}, % 本科生可选,校外指导教师姓名,不用时请留空为{}
%   eassistsupervisor = \secretize{}, % 本科生可选,助理指导教师或/和校外指导教师姓名拼音,不用时请留空为{}
%   cpsupervisor = \secretize{}, % 仅专硕,兼职导师姓名
%   epsupervisor = \secretize{},	% 仅专硕,兼职导师姓名拼音
%   cclass = \secretize{\rmfamily{2023}\heiti{年}\rmfamily{6}\heiti{月}},	% 博士生和学硕填学科门类,学硕填学科类型
%   research_direction = \zihao{3}{工学},
%   edgree = {},	% 专硕填Professional Degree,其他按实情填写
% % 提示:如果内容太长,可以用\zihao{}命令控制字号,作用范围:{}内
%   cmajor = 工~~~~学,	% 专硕不需填,填写专业名称
%   emajor = , % % 专硕不需填,填写专业英文名称
%   cmajora = \zihao{3}{软件工程},
%   cmajorb = \zihao{3}{计算机视觉},
  % cmajorc = \secretize{},
  % cmajord = 2024年6月,
% ===================
% 底部的学院名称和日期
% ===================
  % cdepartment = ,	%学院名称
  % edepartment = ,	%学院英文名称
% ===================
% 封面的日期可以自动生成(注释掉时),也可以解除注释手动指定,例如:二〇一六年五月
% ===================
%	mycdate = {2023年6月},
%	myedate = {June 2023},
}% End of \cqusetup
% ===================
%
% 论文的摘要
%
% ===================
\begin{cabstract}	% 中文摘要


\end{cabstract}
% 中文关键词,请使用英文逗号分隔:
\ckeywords{}

\begin{eabstract}	% 英文摘要

 
\end{eabstract}
% 英文关键词,请使用英文逗号分隔,关键词内可以空格:
\ekeywords{
}

% 封面和摘要配置完成
% 封面部分
% \makecover

\frontmatter %%%前置部分(封面后绪论前)
%\cquauthpage[contents/cover1.pdf]
%\cquauthpage[contents/cover2.pdf]
%\cquauthpage[contents/cover3.pdf]
%\cquauthpage[contents/cover4.pdf]

%% 原创声明和授权说明书,可选:用扫描页替换
%\cquauthpage[authscan.pdf]
%\cquauthpage

% 摘要
\makeabstract

%% 目录,注意需要多次编译才能更新
\setlength{\cftbeforetoctitleskip}{0pt}
\setlength{\cftaftertoctitleskip}{20pt}
\tableofcontents


% \setlength{\cftbeforelottitleskip}{0pt}
% \setlength{\cftafterlottitleskip}{20pt}
%% 插图索引,可选,如不用可注释掉
\renewcommand*{\listfigurename}{图目录}
\clearpage
\phantomsection
\addcontentsline{toc}{chapter}{图目录}
\listoffigures
%\listoffiguresEN
% \setlength{\cftbeforelottitleskip}{0pt}
% \setlength{\cftafterlottitleskip}{20pt}
%% 表格索引,可选
\renewcommand*{\listtablename}{表目录}
\clearpage
\phantomsection
\addcontentsline{toc}{chapter}{表目录}
\listoftables
%\listoftablesEN
%% 公式索引,可选
%\listofequations
%\listofequationsEN
%% 符号对照表,可选
% \clearpage
% \phantomsection 
% \addcontentsline{toc}{chapter}{主要符号对照表}
% \input{contents/denotation}
% %% 缩略语对照表,可选
% \clearpage
% \phantomsection 
% \addcontentsline{toc}{chapter}{缩略语对照表}
% \input{contents/abbreviate}

\mainmatter %%% 主体部分(绪论开始,结论为止)
%* 子文件的多少和内容由你决定(最好以章为单位),基本原则是提速预览、脉络清晰、管理容易。

% 设置字号为小四
\renewcommand{\normalsize}{\fontsize{12pt}{20pt}\selectfont}
% 设置小四正文行间距为 20 磅
\setstretch{1.312}

\chapter[\hspace{0pt}绪\hskip\ccwd{}论]{{\heiti\zihao{3}\hspace{0pt}绪\hskip\ccwd{}论}}\label{chapter: 绪论}

本章内容共分为四节,\hyperref[section1: 研究背景及意义]{第一节}介绍本文的研究背景及意义;\hyperref[section1: 国内外研究现状与挑战]{第二节}总结少样本分类算法的国内外研究现状,并对其面临的挑战进行分析;\hyperref[section1: 本文研究内容与创新点]{第三节}介绍本文的研究内容与创新点;\hyperref[section1: 本文组织结构]{第四节}对本文组织结构进行概括。

\section[\hspace{-2pt}研究背景及意义]{{\heiti\zihao{-3} \hspace{-8pt}研究背景及意义}}\label{section1: 研究背景及意义}

\cite{AFHN}
\include{contents/related_work}
\include{contents/MGRCL}
\include{contents/S2V}
\chapter[\hspace{0pt}总结与未来展望]{{\heiti\zihao{3}\hspace{0pt}总结与未来展望}}\label{section 7}
\removelofgap
\removelotgap

本章内容共分为两节,\hyperref[section5: 总结]{第一节}对本文研究内容与方法进行总结;\hyperref[section5: 未来展望]{第二节}介绍本文所提方法的局限性并对未来研究方向进行展望。

\section[\hspace{-2pt}总结]{{\heiti\zihao{-3} \hspace{-8pt}总结}}\label{section5: 总结}


\section[\hspace{-2pt}未来展望]{{\heiti\zihao{-3} \hspace{-8pt}未来展望}}\label{section5: 未来展望}

%\include{contents/yourFreeChoise}

\backmatter %%% 后置部分(致谢、参考文献、附录等)

%% 参考文献
% 顺序编码制:cqunumerical		
% 注意:至少需要引用一篇参考文献,否则下面两行会引起编译错误。
% \bibliographystyle{cqunumerical}
\bibliographystyle{gbt7714-numerical_new}
\bibliography{ref/refs}


%% 附录(按ABC...分节,证明、推导、程序、个人简历等)
\appendix
% \chapter[附\hskip\ccwd{}\hskip\ccwd{}录]{{\heiti\zihao{3}附\hskip\ccwd{}\hskip\ccwd{}录}}
\section[\hspace{-2pt}作者在攻读学位期间的论文目录]{{\heiti\zihao{-3} \hspace{-8pt}作者在攻读学位期间的论文目录}}

%下面是盲审标记\cs{secretize}的用法,记得去\textsf{main.tex}开启盲审开关看效果:

\circled{1}已发表论文

\begin{enumerate}
    \item \textbf{\secretize{XU X}}, \secretize{LIU K}, DAI P, et al. Joint task offloading and resource optimization in NOMA-based vehicular edge computing: A game-theoretic DRL approach[J]. Journal of Systems Architecture, 2023, 134: 102780. 影响因子: 5.836(2021), 4.497(5年) (中科院SCI 2区,对应本文第三章)
	\item \textbf{\secretize{许新操}}, \secretize{刘凯}, 刘春晖, 等. 基于势博弈的车载边缘计算信道分配方法[J]. 电子学报, 2021,49(5): 851-860. (EI 索引,CCF T1类中文高质量科技期刊,对应本文第三章)
	\item \textbf{ \secretize{XU X}}, \secretize{LIU K}, XIAO K, et al. Vehicular fog computing enabled real-time collision warning via trajectory calibration[J]. Mobile Networks and Applications, 2020, 25(6): 2482-2494. 影响因子: 3.077(2021), 2.92(5年) (中科院SCI 3区,对应本文第五章)
	\item \secretize{LIU K}, \textbf{\secretize{XU X}}, CHEN M, et al. A hierarchical architecture for the future Internet of Vehicles[J]. IEEE Communications Magazine, 2019, 57(7): 41-47. 影响因子: 9.03(2021), 10.892(5年) (中科院SCI 1区,对应本文第二章)
	\item \textbf{ \secretize{XU X}}, \secretize{LIU K}, ZHANG Q, et al. Age of view: A new metric for evaluating heterogeneous information fusion in vehicular cyber-physical systems[C]. Proceedings of IEEE International Conference on Intelligent Transportation Systems (IEEE ITSC’22), Macau, China, October 8-12, 2022. (EI 索引)
	\item \textbf{\secretize{许新操}}, 周易, \secretize{刘凯}, 等. 车载雾计算环境中基于势博弈的分布式信道分配[C]. 第十四届中国物联网学术会议(CWSN’20), 中国敦煌, 2020/9/18-9/21.
	\item \textbf{\secretize{XU X}}, \secretize{LIU K}, XIAO K, et al. Design and implementation of a fog computing based collision warning system in VANETs[C]. Proceedings of IEEE International Symposium on Product Compliance Engineering-Asia (IEEE ISPCE-CN’18), Hong Kong/Shengzhen, December 5-7, 2018. (EI 索引)
	\item LIU C, \secretize{LIU K}, REN H, \textbf{\secretize{XU X}}, et al. RtDS: Real-time distributed strategy for multi-period task offloading in vehicular edge computing environment[J]. Neural Computing and Applications, to appear. 影响因子: 5.102(2021), 5.13(5年) (中科院SCI 2区)
	\item XIAO K, \secretize{LIU K}, \textbf{\secretize{XU X}}, et al. Cooperative coding and caching scheduling via binary particle swarm optimization in software defined vehicular networks[J]. Neural Computing and Applications, 2021, 33(5): 1467-1478. 影响因子: 5.102(2021), 5.13(5年) (中科院SCI 2区)
	\item XIAO K, \secretize{LIU K}, \textbf{\secretize{XU X}}, et al. Efficient fog-assisted heterogeneous data services in software defined VANETs[J]. Journal of Ambient Intelligence and Humanized Computing, 2021, 12(1): 261-273. 影响因子: 3.662 (2021), 3.718 (5年) (中科院SCI 3区)
	\item LIU C, \secretize{LIU K}, \textbf{\secretize{XU X}}, et al. Real-time task offloading for data and computation intensive services in vehicular fog computing environments[C]. Proceedings of IEEE International Conference on Mobility, Sensing and Networking (IEEE MSN’20), Tokyo, Japan, December 17-19, 2020. (EI 索引,CCF C类国际会议)
	\item ZHOU Y, \secretize{LIU K}, \textbf{ \secretize{XU X}}, et al. Multi-period distributed delay-sensitive tasks offloading in a two-layer vehicular fog computing architecture[C]. Proceedings of International Conference on Neural Computing and Applications (NCAA’20), Shenzhen, China, July 3-6, 2020. (EI 索引)
	\item ZHOU Y, \secretize{LIU K}, \textbf{ \secretize{XU X}}, et al. Distributed scheduling for time-critical tasks in a two-layer vehicular fog computing architecture[C]. Proceedings of IEEE Consumer Communications and Networking Conference (IEEE CCNC’20), Las Vegas, USA, January 11-14, 2020. (EI 索引)
\end{enumerate}

\circled{2}已投稿论文

\begin{enumerate}
	\item \textbf{\secretize{XU X}}, \secretize{LIU K}, DAI P, et al. Cooperative sensing and heterogeneous information fusion in VCPS: A multi-agent deep reinforcement learning approach[J]. IEEE Transactions on Intelligent Transportation Systems, under major revision. 影响因子: 9.551 (2021), 9.502 (5年) (中科院SCI 1区,对应本文第二章)
	\item \secretize{LIU K},\textbf{\secretize{XU X}}, DAI P, et al. Cooperative sensing and uploading for quality-cost tradeoff of digital twins in VEC[J]. IEEE Transactions on Consumer Electronics, under minor revision. 影响因子: 4.414 (2021), 3.565 (5年) (中科院SCI 2区,对应本文第四章) 
\end{enumerate}

\section[\hspace{-2pt}作者在攻读学位期间取得的科研成果目录]{{\heiti\zihao{-3} \hspace{-8pt}作者在攻读学位期间取得的科研成果目录}}
\begin{enumerate}
	\item \textbf{\secretize{许新操}}, \secretize{刘凯}, 李东. 一种针对软件定义车联网的控制平面视图构建方法. 发明专利. ZL202110591822.1.
	\item \secretize{刘凯}, 张浪, \textbf{\secretize{许新操}}, 任华玲, 周易. 一种基于边缘计算的盲区车辆碰撞预警方法. 发明专利. ZL201910418745.2.
	\item 任华玲, \secretize{刘凯}, 陈梦良, 周易, \textbf{\secretize{许新操}}. 一种基于雾计算的信息采集、计算、传输架构. 发明专利. ZL201910146357.3.
\end{enumerate}

\section[\hspace{-2pt}作者在攻读学位期间参与的科研项目目录]{{\heiti\zihao{-3} \hspace{-8pt}作者在攻读学位期间参与的科研项目目录}}
\begin{enumerate}
	\item 国家自然科学基金面上项目,面向车联网边缘智能的计算模型部署与协同跨域优化,项目编号: 62172064,2022/01–2025/12.(项目参与人员)
	\item 国家自然科学基金面上项目,面向大规模数据服务的异构融合车联网架构与协议研究,项目编号: 61872049,2019/01–2022/12.(项目参与人员)
\end{enumerate}

\section[\hspace{-2pt}学位论文相关代码]{{\heiti\zihao{-3} \hspace{-8pt}学位论文相关代码}}
\begin{enumerate}
	\item 基于差分奖励的多智能体深度强化学习源代码\\https://github.com/neardws/Multi-Agent-Deep-Reinforcement-Learning
	\item 基于博弈理论的多智能体深度强化学习源代码\\https://github.com/neardws/Game-Theoretic-Deep-Reinforcement-Learning
	\item 基于多目标的多智能体深度强化学习源代码\\https://github.com/neardws/MAMO-Deep-Reinforcement-Learning
	\item 基于车载信息物理融合系统优化的碰撞预警源代码\\https://github.com/neardws/fog-computing-based-collision-warning-system
	\item 基于C-V2X通信的碰撞预警原型系统源代码\\https://github.com/neardws/V2X-based-Collision-Warning
	\item 基于DSRC通信的碰撞预警原型系统源代码\\https://github.com/cqu-bdsc/Collision-Warning-System
	\item 滴滴 GAIA 数据集处理源代码\\https://github.com/neardws/Vehicular-Trajectories-Processing-for-Didi-Open-Data
\end{enumerate}

\newpage
\section[\hspace{-2pt}学位论文数据集]{{\heiti\zihao{-3} \hspace{-8pt}学位论文数据集}}

\begin{table}[h]
\resizebox{\columnwidth}{!}{%
\begin{tabular}{|cllcclclcl|}
\hline
\multicolumn{4}{|c|}{\heiti{关键词}}             & \multicolumn{2}{c|}{\heiti{密级}}   & \multicolumn{4}{c|}{\heiti{中图分类号}}                                    \\ \hline
\multicolumn{4}{|c|}{\begin{tabular}[c]{@{}c@{}}车载信息物理融合系统;\\异构车联网; 车载边缘计算;\\资源优化; 多智能体深度强化学习\end{tabular}} & \multicolumn{2}{c|}{公开} & \multicolumn{4}{c|}{TP} \\ \hline
\multicolumn{3}{|c|}{\heiti{学位授予单位名称}} & \multicolumn{3}{c|}{\heiti{学位授予单位代码}}    & \multicolumn{2}{c|}{\heiti{学位类别}}  & \multicolumn{2}{c|}{\heiti{学位级别}}        \\ \hline
\multicolumn{3}{|c|}{\secretize{重庆大学}}     & \multicolumn{3}{c|}{\secretize{10611}}       & \multicolumn{2}{c|}{学术学位}  & \multicolumn{2}{c|}{博士}          \\ \hline
\multicolumn{4}{|c|}{\heiti{论文题名}}            & \multicolumn{2}{c|}{\heiti{并列题名}} & \multicolumn{4}{c|}{\heiti{论文语种}}                                     \\ \hline
\multicolumn{4}{|c|}{\begin{tabular}[c]{@{}c@{}}车载信息物理融合系统建模与优化关键技术研究\end{tabular}}               & \multicolumn{2}{c|}{无}   & \multicolumn{4}{c|}{中文} \\ \hline
\multicolumn{3}{|c|}{\heiti{作者姓名}}     & \multicolumn{3}{c|}{\secretize{许新操}}         & \multicolumn{2}{c|}{\heiti{学号}}    & \multicolumn{2}{c|}{\secretize{20191401452}} \\ \hline
\multicolumn{6}{|c|}{\heiti{培养单位名称}}                                      & \multicolumn{4}{c|}{\heiti{培养单位代码}}                                   \\ \hline
\multicolumn{6}{|c|}{\secretize{重庆大学}}                                        & \multicolumn{4}{c|}{\secretize{10611}}                                    \\ \hline
\multicolumn{3}{|c|}{\heiti{学科专业}}     & \multicolumn{3}{c|}{\heiti{研究方向}}        & \multicolumn{2}{c|}{\heiti{学制}}    & \multicolumn{2}{c|}{\heiti{学位授予年}}       \\ \hline
\multicolumn{3}{|c|}{计算机科学与技术} & \multicolumn{3}{c|}{车联网}         & \multicolumn{2}{c|}{4年}     & \multicolumn{2}{c|}{\secretize{2023年}}        \\ \hline
\multicolumn{3}{|c|}{\heiti{论文提交日期}}   & \multicolumn{3}{c|}{\secretize{2023年6月}}     & \multicolumn{2}{c|}{\heiti{论文总页数}} & \multicolumn{2}{c|}{\pageref{LastPage}}         \\ \hline
\multicolumn{3}{|c|}{\heiti{导师姓名}}     & \multicolumn{3}{c|}{\secretize{刘凯}}          & \multicolumn{2}{c|}{\heiti{职称}}    & \multicolumn{2}{c|}{教授}          \\ \hline
\multicolumn{6}{|c|}{\heiti{答辩委员会主席}}                                     & \multicolumn{4}{c|}{\secretize{雒江涛}}                                      \\ \hline
\multicolumn{10}{|c|}{\heiti{\begin{tabular}[c]{@{}c@{}} 电子版论文提交格式\\ 文本(\checkmark) 图像() 视频()音频()多媒体()其他()\end{tabular}}}                              \\ \hline
\end{tabular}%
}
\end{table}


%% 致谢
% \chapter[致\hskip\ccwd{}\hskip\ccwd{}谢]{{\heiti\zihao{3}致\hskip\ccwd{}\hskip\ccwd{}谢}}

% 这里用盲审环境包裹致谢,在开启盲审开关时,环境内部的内容不予渲染。
\begin{secretizeEnv}

提笔之际,已到了在重庆生活学习的第六个年头,而我的博士研究生学习阶段也可以说算是告一段落。回想读博期间一路走来,其中有欣喜,也有难过;有深深的孤独,也有现在的恋恋不舍。如今终于到了要道别的时候,所以想借此机会给每一个支持和帮助我的人们好好说一声感谢与有缘再见。

首先,我要衷心感谢我的导师刘凯教授。您是我学术道路上的引路人,您的悉心指导对我产生了巨大影响。您的专业知识、学术见解和研究激情都激发了我不断超越自我的动力。您耐心地解答我的问题,指导我的实验,并对我的论文提出宝贵的建议。您对我的信任和鼓励使我更加自信地迈向学术领域的新阶段。我将永远铭记您对我的慷慨付出和关心。

其次,我要感谢西南交通大学戴朋林老师、重庆邮电大学张浩老师、重庆师范大学肖颗老师、重庆大学国家卓越工程师学院李楚照老师,以及实验室廖成武、金飞宇、任华玲、刘春晖、晏国志、胡峻菠、钟成亮、吴峻源等同学在本学位论文撰写和校对过程中提供的宝贵意见与无私帮助。

再次,我要感谢我的母亲刘菁女士。您生育抚养了我,感谢您对我的无私包容与关爱支持,如果可以,我希望把这篇论文献给您。您是我所见过最坚强的人,都说\qthis{为母则刚},但我也希望能有一天,您能放下心中的重担,为自己好好生活。

此外,我也要感谢实验室的师弟师妹们。在毕业之际,我们一起欢聚于重庆大学国家卓越工程师学院,一起度过了许多个日夜,也为我带来了难忘的回忆。

最后,我要特别感谢答辩主席中国科学院重庆绿色智能技术研究院尚明生教授和所有委员重庆大学郭松涛教授、西北工业大学王柱教授、重庆邮电大学高陈强教授、重庆大学古富强教授的仔细审查和评估。感谢你们在繁忙的工作中抽出时间来对我的研究进行评价,并给予我宝贵的意见和建议。同时,我还要感谢论文评审专家们,你们在匿名评审的过程中,以专业、客观的态度审查了我的论文。你们对我的研究提出的批评和建议,帮助我更好地认识到研究的不足之处,并鼓励我在今后的学术探索中不断进步和改进。衷心感谢各位论文评审专家与答辩委员专家的辛勤工作和付出。

感谢你们陪伴我度过漫长岁月,世界因你们更美好。
\vfill
\begin{flushright}
{\stxingkai \Large 许新操} \hspace*{3.5em}
\\  \hspace*{\fill} \\
{二〇二三年五月\hspace*{1em}于重庆}
\end{flushright}
\end{secretizeEnv}

\end{document}