\cqusetup{
%	************	注意	************
%	* 1. \cqusetup{}中不能出现全空的行,如果需要全空行请在行首注释
%	* 2. 不需要的配置信息可以放心地坐视不理、留空、删除或注释(都不会有影响)
%	*
%	********************************
% ===================
%	论文的中英文题目
% ===================
  ctitle = {基于多元关系建模的少样本分类算法研究},
  etitle = {Research on Few-Shot Classification Algorithms Based on Multivariate Relationship Modeling},
% ===================
% 作者部分的信息
% \secretize{}为盲审标记点,在打开盲审开关时内容会自动被替换为***输出,盲审开关默认关闭
% ===================
  cauthor = \secretize{尹国伟},	% 你的姓名,以下每项都以英文逗号结束
%   eauthor = \secretize{Guowei~Yin},	% 姓名拼音,~代表不会断行的空格
%   studentid = \secretize{},	% 仅本科生,学号
%   csupervisor = \secretize{黄~~~晟~~~~~教授},	% 导师的姓名
%   esupervisor = \secretize{{Prof.~Sheng Huang}},	% 导师的姓名拼音
%   cassistsupervisor = \secretize{}, % 本科生可选,助理指导教师姓名,不用时请留空为{}
%   cextrasupervisor = \secretize{}, % 本科生可选,校外指导教师姓名,不用时请留空为{}
%   eassistsupervisor = \secretize{}, % 本科生可选,助理指导教师或/和校外指导教师姓名拼音,不用时请留空为{}
%   cpsupervisor = \secretize{}, % 仅专硕,兼职导师姓名
%   epsupervisor = \secretize{},	% 仅专硕,兼职导师姓名拼音
%   cclass = \secretize{\rmfamily{2023}\heiti{年}\rmfamily{6}\heiti{月}},	% 博士生和学硕填学科门类,学硕填学科类型
%   research_direction = \zihao{3}{工学},
%   edgree = {},	% 专硕填Professional Degree,其他按实情填写
% % 提示:如果内容太长,可以用\zihao{}命令控制字号,作用范围:{}内
%   cmajor = 工~~~~学,	% 专硕不需填,填写专业名称
%   emajor = , % % 专硕不需填,填写专业英文名称
%   cmajora = \zihao{3}{软件工程},
%   cmajorb = \zihao{3}{计算机视觉},
  % cmajorc = \secretize{},
  % cmajord = 2024年6月,
% ===================
% 底部的学院名称和日期
% ===================
  % cdepartment = ,	%学院名称
  % edepartment = ,	%学院英文名称
% ===================
% 封面的日期可以自动生成(注释掉时),也可以解除注释手动指定,例如:二〇一六年五月
% ===================
%	mycdate = {2023年6月},
%	myedate = {June 2023},
}% End of \cqusetup
% ===================
%
% 论文的摘要
%
% ===================
\begin{cabstract}	% 中文摘要


\end{cabstract}
% 中文关键词,请使用英文逗号分隔:
\ckeywords{}

\begin{eabstract}	% 英文摘要

 
\end{eabstract}
% 英文关键词,请使用英文逗号分隔,关键词内可以空格:
\ekeywords{
}

% 封面和摘要配置完成